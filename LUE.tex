\documentclass{article}
\usepackage{enumitem}
\usepackage{amsfonts}
\usepackage{amsmath}
\usepackage{amstext}
\usepackage{amsthm}
\usepackage{graphicx}
\usepackage[braket,qm]{qcircuit}
\usepackage{algorithm}
\DeclareMathOperator{\Tr}{Tr}
\newtheorem{theorem}{Theorem}
\newtheorem{proposition}{Proposition}
\usepackage[noend]{algpseudocode}
\usepackage{hyperref}
\title{On local unitary equivalence of two qudits states}
\author{Guangliang He
  \thanks{\href{mailto:guangliang.he@gmail.com}
    {Email: guangliang.he@gmail.com}}}
\date{\today}

\begin{document}
\maketitle

\section{Introduction}
Entanglement, ``which is considered to be the most nonclassical
manifestations of quantum formalism''\cite{Horodecki_2009}
is attracting more and more attention in the science of quantum
information and quantum computing.

There is an extensive research about understanding entanglement,
analysing the structure entanglement states, in particular,
analysing the local unitary equivalence (LUE) of two quantum states.
Li et al.\cite{Li_2014} and Martins\cite{Martins_2015} presented
test criteria for LUE of multipartite qubit states.
Zhou etal.\cite{Zhou_2024} developed a set of LU invariants for
the necesary condition for LUE between two multipartite qudit states.

In this work, we present a systematic process for testing
necessary and sufficient condition of local unitary equivalence
between two multipartite qudit states.

The rest of this paper is organized as follows. In
Section~\ref{sec:setup}, we state the problem and introduce the
mathematical notations.  In Section~\ref{sec:lue}, we introduce
and prove the main theorem about local unitary equivalence.
In Section~\ref{sec:finding_theta}, we lay out some important
steps to solve the problem.  In Section~\ref{sec:steps}, we
present a step-by-step procedure to test LUE.  We keep some
tedieous mathematical materials in the Appendix.

\section{Setup}
\label{sec:setup}
Let $\rho^{(a)}$ where $a = \{1, 2\}$  be two density matrices of $K$ qudits
in the Hilbert space $H = H_1\otimes\cdots\otimes H_K$, with dimensions
$N_1,\ldots,N_K$. The two density matrices $\rho^{(1)}$ and $\rho^{(2)}$
are said to be local unitary equivalent (LUE) when there exist a set
of unitary matrices $U_k\in SU(N_k)$ such that
$U = U_1\otimes\cdots\otimes U_K$ and
\begin{equation}
  \label{eq:rho_lue}
  \rho^{(2)} = U\rho^{(1)}U^\dagger.
\end{equation}
For each density matrix, consider its single qudit reduced density matrix
\begin{equation}
  \label{eq:rho_a_k}
  \rho^{(a)}_k = \Tr_{|k}\rho^{(a)},
\end{equation}
here the notation $\Tr_{|k}$ means take partial trace of every subspace
except in $H_k$.  Let $D_k^{(a)}$ and $V_k^{(a)}$ be the eigenvalue and
eigenvectors of $\rho_k^{(a)}$ such that
\begin{equation}
  \rho^{(a)}_k = V^{(a)}_kD^{(a)}_kV^{(a)\dagger}_k.
\end{equation}
Define $V^{(a)}$ as
\begin{equation}
  V^{(a)} = V^{(a)}_1\otimes\cdots\otimes V^{(a)}_K,
\end{equation}
and the reference density matrices as
\begin{equation}
  \label{eq:rho_ref}
  \rho^{(r, a)} = V^{(a)\dagger}\rho^{(a)}V^{(a)}.
\end{equation}
Here we use the superscript $r$ to differentiate the reference density
matrices from the original density matrices. Taking the partial trace,
we have
\begin{equation}
  \label{eq:rho_ref_k}
  \rho^{(r,a)}_k = D^{(a)}_k.
\end{equation}

Let $\{T^{(N_k)}_{i_k}\mid i = 1,\ldots N_k^2-1\}$ be the set of generators
of the defining representation of $\mathfrak{su}(N_k)$\cite{Haber_2021},
and $T^{(N_k)}_0 = I_{N_k}/\sqrt{2N_k}$.
We have
\begin{eqnarray}
  \label{eq:trace_t}
  \Tr T^{(N_k)}_i & = & \delta_{i0}\sqrt{\frac{N_k}2} \\
  \label{eq:trace_t_t}
  \Tr T^{(N_k)}_iT^{(N_k)}_j & = &\frac{\delta_{ij}}2.
\end{eqnarray}
Projecting $\rho^{(r,a)}$ to the basis
$\{T^{(N_1)}_{i_1}\otimes\cdots\otimes T^{(N_K)}_{i_K}\}$,
\begin{equation}
  \label{eq:ref_rho_proj}
  \rho^{(r,a)} = \sum_{i_1=0}^{N_1^2-1}\cdots\sum_{i_K=0}^{N_K^2-1}
  p^{(a)}_{i_1\ldots i_K}T^{(N_1)}_{i_1}\otimes\cdots\otimes T^{(N_K)}_{i_K},
\end{equation}
where
\begin{equation}
  p^{(a)}_{i_1\cdots i_K} = 2^K\Tr\left(\rho^{(r,a)}T^{(N_1)}_{i_1}\otimes\cdots\otimes T^{(N_K)}_{i_K}\right).
\end{equation}

Taking the trace of Eq.~(\ref{eq:ref_rho_proj}) gives us
\begin{equation}
  p^{(a)}_{0\cdots0} = \prod_{k=1}^K\sqrt{\frac{2}{N_k}}.
\end{equation}

Taking the partial trace $\Tr_{|k}$ of Eq.~(\ref{eq:ref_rho_proj}),
\begin{equation}
  \rho^{(r,a)}_k = \Tr_{|k}\rho^{(r,a)} = \frac{I_{N_k}}{N_k}
  + \left(\prod_{i\ne k}\sqrt{\frac{N_i}2}\right)\sum_{i_k=1}^{N_k^2-1}p^{(a)}_{k,i_k}T^{(N_k)}_{i_k}
\end{equation}

\section{Necessary and sufficient condition for LUE}
\label{sec:lue}
We will begin with a simple theorem.
\begin{theorem}
  \label{thm:lue}
  The following three statements are equivalent:
  \begin{enumerate}[label=\textbf{S.\arabic*}]
  \item \label{thm_stm_rho_lue} $\rho^{(1)}$ and $\rho^{(2)}$ are local unitary equivalent.
  \item \label{thm_stm_ubar} There exists a set of real vectors
    $\{\theta^{(k)}\mid k = 1,\ldots,K\}$ such that
    \begin{equation}
      \label{eq:ref_rho_lue}
      \rho^{(r,2)} = \left(\bigotimes_{k=1}^K \bar U_k\right)
      \rho^{(r,1)}
      \left(\bigotimes_{k=1}^K \bar U_k\right)^\dagger,
    \end{equation}
    where
    \begin{equation}
      \label{eq:ubar_k_exp_map}
      \bar U_k = \exp\left(i\sum_{i=1}^{N_k^2-1}\theta^{(k)}_iT^{(N_k)}_i\right).
    \end{equation}
  \item \label{thm_stm_R} There exists a set of real vectors
    $\{\theta^{(k)}\mid k = 1,\ldots,K\}$ such that
    \begin{equation}
      p^{(2)}_{i_1\cdots i_K} = \sum_{j_1=0}^{N_1^2-1}\cdots\sum_{j_K=0}^{N_K^2-1}
      \tilde R^{(1)}_{i_1j_1}\cdots\tilde R^{(K)}_{i_Kj_K}
      p^{(1)}_{j_1\cdots j_K},
    \end{equation}
    where
    \begin{equation}
      \label{eq:tilde_R_k}
      \tilde R^{(k)} = \begin{bmatrix}1 & 0 \\
        0 & R^{(k)}
        \end{bmatrix},
    \end{equation}
    and
    \begin{equation}
      \label{eq:R_k}
      R^{(k)} = \exp\left(i\sum_{i=1}^{N_k^2-1}\theta^{(k)}_iF^{(N_k)}_i\right),
    \end{equation}
    with $F^{(k)}_i$ the adjoint representation of $T^{(N_k)}_i$.
  \end{enumerate}
\end{theorem}
\begin{proof}  We will first prove from \ref{thm_stm_rho_lue} to
  \ref{thm_stm_ubar}.  By the definition of $\rho^{(1)}$ and $\rho^{(2)}$
  are local unitary equivalent, these exists a set of
  $U_k\in SU(N_k)$ for $k = 1,\ldots,K$ such that
  \begin{equation}
    \rho^{(2)} = \left(\bigotimes_{k=1}^KU_k\right)
    \rho^{(1)}
    \left(\bigotimes_{k=1}^KU_k\right)^\dagger.
  \end{equation}
  Let $\bar U_k = V^{(2)\dagger}_kU_kV^{(1)}_k$, the definition of $\rho^{(r,a)}$
  leads us directly to Eq.~(\ref{eq:ref_rho_lue}).  The fact that $U_k$,
  $V^{(a)}_k$ are all in $SU(N_k)$ leads to $\bar U_k\in SU(N_k)$ and
  Eq.~(\ref{eq:ubar_k_exp_map}) follows directly.
  
  Now we will prove from \ref{thm_stm_ubar} to \ref{thm_stm_R}.
  From \ref{thm_stm_ubar} and Eq.~(\ref{eq:ref_rho_proj}), we have
  \begin{eqnarray}
    & & \sum_{i_1=0}^{N_1^2-1}\cdots\sum_{i_K=0}^{N_K^2-1}p^{(2)}_{i_1\ldots i_K}
    T^{(N_1)}_{i_1}\otimes\cdots\otimes T^{(N_K)}_{i_K} \nonumber \\ 
    \label{eq:proj_ref}
    & = & \sum_{i_1=0}^{N_1^2-1}\cdots\sum_{i_K=0}^{N_K^2-1}p^{(1)}_{i_1\ldots i_K}
    \left(\bar U_1T^{(N_1)}_{i_1}\bar U_1^\dagger\right)\otimes\cdots\otimes
    \left(\bar U_KT^{(N_K)}_{i_K}\bar U_K^\dagger\right)
  \end{eqnarray}
  As shown in Appendix~\ref{appendix:adj_rep}, for $1 \le i_k \le N_k^2-1$,
  \begin{equation}
    \bar U_kT^{(N_k)}_{i_k}\bar U_k^\dagger =
    \sum_{j_k=1}^{N_k^2-1}T^{(N_k)}_{j_k}R^{(k)}_{j_ki_k},
  \end{equation}
  where $R^{(k)}$, defined as in Eq.~(\ref{eq:R_k}),
  is the adjoint representation of $\bar U_k$ and $F^{(N_k)}_{i_k}$ is the
  adjoint representation of $T^{(N_k)}_{i_k}$.  Together with
  $\bar U_kT^{(N_k)}_0\bar U_k^\dagger = T^{(N_k)}_0$, we have
  \begin{equation}
    \bar U_kT^{(N_k)}_{i_k}\bar U_k^\dagger =
    \sum_{j_k=0}^{N_k^2-1}T^{(N_k)}_{j_k}\tilde R^{(k)}_{j_ki_k},
    \qquad 0 \le i_k \le N_k^2-1,
  \end{equation}
  with $\tilde R^{(k)}$ as defined in Eq.~(\ref{eq:tilde_R_k}).

  Finally we complete the proof by showing \ref{thm_stm_R} leads to
  \ref{thm_stm_rho_lue}.  If \ref{thm_stm_R} is true, let
  $\bar U_k = \exp\left(i\sum_{i_k=1}^{N_k^2-1}\theta^{(k)}_{i_k}T^{(N_k)}_{i_k}\right)$,
  then $R^{(k)}$ is the adjoint representation of $\bar U_k$,
  which leads to Eq.~(\ref{eq:ref_rho_lue}).  Combining with
  Eq.~(\ref{eq:rho_ref}), we arrive with the conclusion of
  \ref{thm_stm_rho_lue}.
\end{proof}


\section{Finding \texorpdfstring{$\{\theta^{(k)}\}$}{theta.k}}
\label{sec:finding_theta}
From Theorem~\ref{thm:lue}, the problem of testing local unitary
equivalence between $\rho^{(1)}$ and $\rho^{(2)}$ becomes testing
the existance of the vector set $\{\theta^{(k)}\mid k = 1,\ldots,K\}$.

If $\rho^{(1)}$ and $\rho^{(2)}$ are local unitary equivalent, take
partial trace $\Tr_{|k}$ on both sides of Eq.~(\ref{eq:ref_rho_lue}),
and combine with Eq.~(\ref{eq:rho_ref_k}), we have
\begin{equation}
  D_k = \bar U_k D_k\bar U_k^\dagger.
\end{equation}
This result is a direct generalization of Proposition~1 of
Martins\cite{Martins_2015}.

Let $\tilde D_k = D_k-I_{N_k}/N_k$ and $\bar U_k = \exp(i\bar u_k)$, then
$\tilde D_k\in \mathfrak{su}(N_k)$ and
$\bar u_k = \sum_{j=1}^{N_k^2-1}\theta^{(k)}_jT^{(N_k)}_j\in
\text{C}_{\mathfrak{su}(N_k)}(\tilde D_k)$ where
$\text{C}_{\mathfrak{su}(N_k)}(\tilde D_k)$
is the centralizer of $\tilde D_k$.  As shown in
Appendix~\ref{appendix:centralizer}, this implies a set of constraints to
$\theta^{(k)}$,
\begin{equation}
  \label{eq:theta_con}
  M_k\theta^{(k)} = 0.
\end{equation}
Selecting the $SU(N)$ generators as in Eq.~(\ref{eq:t_a}) and
Eq.~(\ref{eq:t_a_ordering}), the matrix $M_k$ has the form
\begin{equation}
  \label{eq:M_k}
  M_k = \left(\bigoplus_{1\le i<j\le N}(d_{k,j}Y-d_{k,i}Y)\right)\oplus0_{N-1}.
\end{equation}
For each pair $1 \le i < j \le N_k$ that $d_{k,i} \ne d_{k,j}$, there
are a pair of constraint on $\theta^{(k)}$ such that
\begin{equation}
  \label{eq:theta_constraint}
  \theta^{(k)}_{0ij} = \theta^{(k)}_{1ij} = 0.
\end{equation}

Now let us consider the case when all diagonal entries in $D_k$ are distinct.
This means for all
$1 \le i < j \le N$, $d_{k,i}-d_{k,j} = 0.$  Eq.~(\ref{eq:theta_constraint})
tells us that $\theta^{(k)}_{0ij} = \theta^{(k)}_{1ij} = 0$.  Only possible
nonzero in $\theta^{(k)}$ are the ones corresponding to diagonal generators
$T_l$.  In this case, $\bar U_k$ must be diagonal, Eq.~(\ref{eq:R_g_diag})
in Appendix~\ref{appendix:adj_rep} shows that
\begin{equation}
  \label{eq:R_type_A}
  R^{(k)} = \left(\bigoplus_{1\le i < j \le N}
  \begin{bmatrix}
    \cos(\eta_i-\eta_j) & \sin(\eta_i-\eta_j) \\
    -\sin(\eta_i-\eta_j) & \cos(\eta_i-\eta_j)
  \end{bmatrix}\right)
  \oplus I_{N-1}.
\end{equation}
For our convenience, we call this type qudits type A qudits.

A type B qudit has some degeneracy in $D_k$,
but with at least one $d_{k,i}$ distinct from all others.
We order the eigenvalues in such a way so $D_{k,N}$ is distinct from
all others.  Thus $\theta^{(k)}_{0iN} = \theta^{(k)}_{1iN} = 0$ for all $i$.
On the other hand, by choosing the generators as the generalized Gell-Mann
matrices divided by 2, we have
\begin{equation}
  [T_{N-1},\, T_{0ij}] = [T_{N-1},\, T_{1ij}] = 0,\qquad 1\le i < j \le N-1.
\end{equation}
Eq.~(\ref{eq:R_g_1}) in Appendix~\ref{appendix:adj_rep} means
\begin{equation}
  \label{eq:R_type_B}
  R^{(k)} = S^{(k)}\oplus1,
\end{equation}
where $S^{(k)}$ is a $(N^2-2)\times(N^2-2)$ matrix.

Now consider two qudits $k_1$ and $k_2$
Let $P^{(a,k_1k_2)}$, $a = 1, 2$ be two matrices defined by
\begin{equation}
  \left(P^{(a,k_1k_2)}\right)_{i_{k_1}i_{k_2}} = p^{(a)}_{0\cdots0i_{k_1}0\cdots0i_{k_2}0\cdots0},
\end{equation}
where $1 \le i_{k_1} \le N_{k_1}^2-1$, $1 \le i_{k_2} \le N_{k_2}^2-1$.
According to Theorem~\ref{thm:lue} \ref{thm_stm_R}, LUE means
\begin{equation}
  \label{eq:RP}
  P^{(2,k_1k_2)} = R^{(k_1)}P^{(1,k_1k_2)}(R^{(k_2)})^T.
\end{equation}
Assume qudit $k_1$ is type A and $k_2$ either type A or B, then
$R^{(k_1)}$ has the form as in Eq.~(\ref{eq:R_type_A}) and
$R^{(k_2)}$ as in Eq.~(\ref{eq:R_type_B}).  This leads to
\begin{equation}
  \label{eq:vRv}
  v^{(2,k_1k_2)} =  R^{(k_1)}v^{(1,k_1k_2)},
\end{equation}
where $v^{(a,k_1k_2)}$ is the last column of matrix $P^{(a,k_1k_2)}$
for $a = 1, 2$.

$R^{(k_1)}$ and $\theta^{(k_1)}$ can be solved following the steps as in
Appendix~\ref{appendix:diag_U}.

For any type qudit $k_1$, if $R^{(k_2)}$ is known, Eq.~(\ref{eq:RP})
tells us that
\begin{equation}
  \label{eq:k1k2}
  P^{(2,k_1k_2)}R^{(k_2)} = R^{(k_1)}P^{(1,k_1k_2)}.
\end{equation}
If $P^{(1,k_1k_2)}$ is invertible, then
\begin{equation}
  R^{(k_1)} = P^{(2,k_1k_2)}R^{(k_2)}(P^{(1,k_1k_2)})^{-1}.
\end{equation}
In the case that $P^{(1,k_1k_2)}$ is not invertible, if we have already solved
other qudits $k_3, k_4, \ldots$, then we have
\begin{equation}
  \label{eq:k1k3}
  P^{(2,k_1k_3)}R^{(k_3)} = R^{(k_1)}P^{(1,k_1k_3)}.
\end{equation}
Continue to add solved qudits until the combined Eq.~(\ref{eq:k1k2}),
Eq.~(\ref{eq:k1k3}), and so on until either having enough ranks to
uniquely solve for $R^{(k_1)}$ or exhausted all solved qudits.

Once $R^{(k_1)}$ is solved, we can follow the steps in
Appendix~\ref{appendix:general_U} to solve for $\theta^{(k_1)}$ and
$\bar U_{k_1}$.

Consider a general type qudit $k_1$ and a type B qudit $k_2$, because
$R^{(k_1)}$ does not have the block diagonal form, Eq.~(\ref{eq:vRv})
is not enough to solve for $R^{(k_1)}$.  If we have other type B
qudits in $k_3,k_4,\ldots$ then
\begin{eqnarray}
  v^{(2,k_1k_3)} & = &  R^{(k_1)}v^{(1,k_1k_3)} \\
  v^{(2,k_1k_4)} & = &  R^{(k_1)}v^{(1,k_1k_4)} \\
              &\cdots&
\end{eqnarray}
If these equations add enough rank to Eq.~(\ref{eq:vRv}), then we
can solve for $R^{(k_1)}$.

One matrix $R^{(k)}$ of the general form for qudit $k$ is uniquely
solved, we can follow the steps in Appendix~\ref{appendix:general_U},
to determine if it is a member of $\text{Ad}(SU(N_{k1}))$, and if so
to solve for $\theta^{(k)}$ and $\bar U_k$.

\section{Complete step by step procedure}
\label{sec:steps}
In this section, we will state the complete step by step procedure
for testing local unitary equivelance of two qudit density matrices.

Let us start with two density matrices $\rho^{(1)}$ and $\rho^{(2)}$
in the Hilbert space $H = H_1\otimes\cdots\otimes H_K$.
\begin{enumerate}
\item For $a = 1, 2$, $k = 1,\ldots,K$, compute single qudit
  reduced density matrices $\rho^{(a)}_k$ as in Eq.~(\ref{eq:rho_a_k}).
\item For each $\rho^{(a)}_k$, compute its eigendecomposition $D^{(a)}_k$
  and $V^{(a)}_k$.
\item For $k = 1,\ldots,K$, test if $D^{(1)}_k = D^{(2)}_k$.  If the test
  fails for any $k$, then $\rho^{(1)}$ and $\rho^{(2)}$ are not local
  unitary equivalent.
\item Build reference reduced density matrices as in
  Eq.~(\ref{eq:rho_ref}).
\item For $k = 1,\ldots,K$, check for degeneracy in $D^{(1)}_k$,
  determine the zero constraints on $\theta^{(k)}$ as in
  Eq.~(\ref{eq:theta_constraint}), and collect the lists of type A
  and type B qudits.
\item For the following case
  \begin{enumerate}
  \item There are two or more type A qudits, or
  \item There are one type A qudit and one or more type B qudits, or
  \item There is no type A qudit but enough type B qudits.
  \end{enumerate}
  Start solving for $R^{(k)}$, $\theta^{(k)}$ following the steps laid
  out in Section~\ref{sec:finding_theta}.  If we find a set of
  $\{\bar U_k\}$ satisfying \ref{thm_stm_ubar} of Theorem~\ref{thm:lue},
  then $\rho^{(1)}$ and $\rho^{(2)}$ are local unitary equivalent.
  If $P^{(1)}$ and $P^{(2)}$ are not consistent, or the solution
  $R^{(k)}$ is not in $\text{Ad}(SU(N_k))$ for any $k$, then
  $\rho^{(1)}$ and $\rho^{(2)}$ are not local unitary equivalent.
\item If not all $\{\theta^{(k)}\}$ can be solved algebraically, then
  solve the minimization problem
  \begin{equation}
    \min_{\theta^{(k)}}\lVert \rho^{(r,2)}-\bar U\rho^{(r,1)}\bar U^\dagger\rVert^2,
  \end{equation}
  where
  \begin{equation}
    \bar U = \bigotimes_{k=1}^K\exp\left(i\sum_{a=1}^{N_k^2-1}
    \theta^{(k)}_aT^{(N_k)}_a\right).
  \end{equation}
  Note that all the solved or partial solved $\theta^{(k)}$ in the
  previous steps are fixed at their solved values, and those zero
  constrained $\theta^{(k)}$ as in Eq.~(\ref{eq:theta_constraint})
  are fixed at zero during the minimization process.
  If the minimum is zero, then $\rho^{(1)}$ and $\rho^{(2)}$ are
  local unitary equivalent, otherwise, they are not.

  A word of caution, numerical minimization may produce local
  minimum, it is prudent to try multiple starting points to
  increase the chance finding the global minimum.
\end{enumerate}

\appendix
\section{Generalized Gell-Mann Matrices}
\label{appendix:ggm}
The definition of the generalized Gell-Mann matrices\cite{Bertlmann_2008}
for $SU(N)$ group can be summarized as the following,
\begin{eqnarray}
  \Lambda_{0ij} & = & \op{i}{j}+\op{j}{i} \\
  \Lambda_{1ij} & = & -i\op{i}{j}+i\op{j}{i} \\
  \Lambda_{l}   & = & \sqrt{\frac{2}{l(l+1)}}\left(
    \sum_{k=1}^l\op{k}{k}-l\op{l+1}{l+1}\right).
\end{eqnarray}
where $1 \le i < j \le N$ and $1 \le l < N$.

The set of generators for $SU(N)$ are chosen as
\begin{equation}
  \label{eq:t_a}
  T_a = \frac{\Lambda_a}{2},
\end{equation}
where index $a$ are in the form of $0ij$, $1ij$, $l$.  We order
the generators as
\begin{equation}
  \label{eq:t_a_ordering}
  \left\{T_{012}, T_{112}, \ldots, T_{0(N-1)N}, T_{1(N-1)N}, T_1,\ldots,
  T_{N-1}\right\}.
\end{equation}
We always use this set of $\{T_a\}$ with this ordering when a specific
set of generators for $SU(N)$ is needed throughout this work.


\section{Adjoint representation}
\label{appendix:adj_rep}
Consider the group $SU(N)$ and its Lie algebr $\mathfrak{su}(N)$, the adjoint
representation\cite{hall2000elementaryintroductiongroupsrepresentations}
$\text{Ad}_g$ of $SU(N)$ is a map
$\mathfrak{su}(N)\rightarrow\mathfrak{su}(N)$ such that $\forall g\in SU(N)$
and $X \in \mathfrak{su}(N)$,
\begin{equation}
  \text{Ad}_g(X) = gXg^{-1}.
\end{equation}
Let $\{T_a\}$ be the set of generators of the defining representation of
$\mathfrak{su}(N)$,
\begin{equation}
  \label{eq:R_g}
  \text{Ad}_g(T_a) = gT_ag^{-1} = \sum_b T_b R_{g,ba},
\end{equation}
Consider the $\exp$ map, $g = \exp\left(i\sum_a\theta_a T_a\right)$, we have
\begin{equation}
  \text{Ad}_g = \exp\left(i\sum_a\theta_a\text{ad}T_a\right)
  = \exp\left(i\sum_a\theta_aF_a\right),
\end{equation}
where $\{F_a\}$ is the set of generators of the adjoint representation
of $\mathfrak{su}(N)$, and
\begin{equation}
  R_g = \exp\left(i\sum_a\theta_aF_a\right).
\end{equation}

If $g$ commutes with a particular generator $T_a$, then $gT_ag^{-1} = T_a$,
Eq.~(\ref{eq:R_g}) yields
\begin{equation}
  \label{eq:R_g_1}
  R_{g,ba} = \delta_{ab}.
\end{equation}

If $g\in SU(N)$ is diagonal, $g = \text{diag}(e^{i\eta_1},\ldots,e^{i\eta_N})$,
with the generators $T_a = \Lambda_a/2$, we can directly compute that
\begin{eqnarray}
  gT_{0ij}g^\dagger & = & \cos(\eta_i-\eta_j)T_{0ij} + \sin(\eta_i-\eta_j)T_{1ij} \\
  gT_{1ij}g^\dagger & = & -\sin(\eta_i-\eta_j)T_{0ij} + \cos(\eta_i-\eta_j)T_{1ij} \\
  gT_lg^\dagger & = & T_l.
\end{eqnarray}
Thus
\begin{equation}
  \label{eq:R_g_diag}
  R_g = \left(\bigoplus_{1\le i < j \le N}
  \begin{bmatrix}
    \cos(\eta_i-\eta_j) & \sin(\eta_i-\eta_j) \\
    -\sin(\eta_i-\eta_j) & \cos(\eta_i-\eta_j)
  \end{bmatrix}\right)
  \oplus I_{N-1}.
\end{equation}

If $H\in \mathfrak{su}(N)$ is diagonal, $H = \text{diag}(H_1,\ldots,H_N)$,
then with $T_a = \Lambda_a/2$, we have
\begin{eqnarray}
  [H,\, T_{0ij}] & = & i(H_i-H_j)T_{1ij}, \\{}
  [H,\, T_{1ij}] & = & -i(H_i-H_j)T_{0ij}, \\{}
  [H,\, T_l] & = & 0.
\end{eqnarray}
That is
\begin{equation}
  \label{eq:ad_h}
  \text{ad}_H = \left(\bigoplus_{i,j}(H_i-H_j)Y\right)\oplus0_{N-1},
\end{equation}
where $Y$ is the second Pauli matrix.


\section{The centralizer of \texorpdfstring{$\mathfrak{su}(N)$}{su(N)}}
\label{appendix:centralizer}
For a Lie algebra $\mathfrak{g}$, the centralizer\cite{jacobson_1979}
of a subset $S$ of $\mathfrak{g}$ is
\begin{equation}
  \text{C}_{\mathfrak{g}}(S) = \{x\in \mathfrak{g}\mid xs = sx \text{ for all } s\in \mathfrak{g}\}
\end{equation}
The notation $\text{C}_{\mathfrak{g}}(s)$ is also used for a singleton set
$S = \{s\}$.

Let $s = \sum_a\sigma_aT_a\in \mathfrak{su}(N)$, its centralizer is
\begin{equation}
  \text{C}_{\mathfrak{su}(N)}(s) = \{x\in \mathfrak{su}(N) \mid xs=sx \text{ for all } s\in S\}
\end{equation}
Write $x$ as
\begin{equation}
  x = \sum_a \chi_aT_a,
\end{equation}
the commuting condition $xs = sx$ is equivalent to
\begin{equation}
 \left[\sum_a\chi_aT_a, \sum_b\sigma_bT_b\right] = 0.
\end{equation}
Define matrix $M$ as
\begin{equation}
  M = \sum_c\sigma_cF_c = \text{ad}_s,
\end{equation}
where $F_c$ is the generators of adjoint representation of $\mathfrak{su}(N)$,
the commuting condition becomes
\begin{equation}
  \label{eq:mx_eq_0}
  Mx = 0,
\end{equation}
where $x$ is a column vector with element $\chi_a$.

Now consider the case $s$ is diagonal, that is
\begin{equation}
  s = \text{diag}(s_1,\ldots,s_N).
\end{equation}
According to Eq.~(\ref{eq:ad_h}), we have
\begin{equation}
  M = \text{ad}_s = \left(\bigoplus_{1\le i < j\le N}(s_i-s_j)Y\right)
  \oplus0_{N-1}.
\end{equation}

\section{Solving for diagonal \texorpdfstring{$U\in SU(N)$}{U in SU.N}}
\label{appendix:diag_U}
Let $U$ be a diagonal element of $SU(N)$, its exponential map is
\begin{equation}
  \label{eq:U_exp_map}
  U = \exp\left(i\sum_{l=1}^{N-1}\theta_lT_l\right),
\end{equation}
where $T_l = \Lambda_l/2$ is diagonal. The adjoint representation of $U$
is
\begin{equation}
  \label{eq:R_exp_map}
  R = \exp\left(i\sum_{l=1}^{N-1}\theta_lF_l\right).
\end{equation}
where $F_l = \text{ad}_{T_l}$ with matrix elements
\begin{equation}
  \left(F_l\right)_{ab} = -if_{lab},
\end{equation}
here $f_{lab}$ is the structure constants of $SU(N)$.  Using the definition
of the generalized Gell-Mann matrices in Appendix~\ref{appendix:ggm}, we
have
\begin{equation}
  F_l = \left(\bigoplus_{1\le i < j\le N}f_{l,0ij,1ij}Y\right)\oplus0_{N-1},
\end{equation}
where $Y$ is the second Pauli matrix and $0_{N-1}$ is a $(N-1)\times(N-1)$
matrix of zeros.  Using the identity
$\exp(i\phi Y) = \cos\phi I+i\sin\phi Y$, we have
\begin{eqnarray}
  R & = & \exp\left(\sum_{l=1}^{N-1}i\theta_lF_l\right) \\
  & = & \exp\left(\left(\bigoplus_{1\le i<j\le N}
  i\sum_{l=1}^{N-1}\theta_lf_{l,0ij,1ij}Y\right)\oplus0_{N-1}\right) \\
  & = & \left(\bigoplus_{1\le i<j\le N}R^{(ij)}\right)\oplus I_{N-1},
\end{eqnarray}
where
\begin{eqnarray}
  R^{(ij)} & = & \cos\phi_{ij}I
  + i\sin\phi_{ij} Y \\
  \label{eq:phi_ij}
  \phi_{ij} & = & \sum_{l=1}^{N-1}\theta_lf_{l,0ij,1ij}
\end{eqnarray}

Assume there are two vectors $v^{(1)}$ and $v^{(2)}$ such that
\begin{equation}
  v^{(2)} = Rv^{(1)}.
\end{equation}
Partition $v^{(a)}$ into $2\times1$ vectors $v^{(a)}_{ij}$ and
$(N-1)\times1$ vector $v^{(a)}_H$ so that
\begin{equation}
  v^{(a)} = \left(\bigoplus_{1\le i < j \le N}v^{(a)}_{ij}\right)\oplus v^{(a)}_H.
\end{equation}
Here the notation $\oplus$ represents concatenation of vectors
(not to be confused with direct sum of matrices).
We have
\begin{eqnarray}
  v^{(2)}_{ij} & = & R^{(ij)}v^{(1)}_{ij}, \\
  v^{(2)}_H & = & v^{(1)}_H.
\end{eqnarray}
For each pair $\lVert v^{(1)}\rVert = \lVert v^{(2)}\rVert > 0$,
one can solve for $\phi_{ij}$ as
\begin{eqnarray}
  \cos\phi_{ij} & = & \frac{v^{(1)}_{ij,1}v^{(2)}_{ij,1}+v^{(1)}_{ij,2}v^{(2)}_{ij,2}}
          {\lVert v^{(1)}_{ij}\rVert^2}, \\
  \sin\phi_{ij} & = & \frac{v^{(1)}_{ij,2}v^{(2)}_{ij,1}-v^{(1)}_{ij,1}v^{(2)}_{ij,2}}
          {\lVert v^{(1)}_{ij}\rVert^2}, \\
  \phi_{ij} & = & \text{atan2}(\sin\phi_{ij}, \cos\phi_{ij}).
\end{eqnarray}
According to Eq.~(\ref{eq:phi_ij}), $\phi_{ij}$ is a linear combination
of $\theta_l$, with the set of solved $\phi_{ij}$, that gives us a system of
linear equation to solve for $\theta$.

If the rank of the system is $N-1$, then we can uniquely solve $\theta$.
Thus $R$ is completely given by Eq.~(\ref{eq:R_exp_map}) and $U$ by
Eq.~(\ref{eq:U_exp_map}).

If the rank is under than $N-1$,
sometime we can solve for part of $\theta$ vector, which would be useful
for our problem, too.

\section{Solving for \texorpdfstring{$U$}{U} from \texorpdfstring{$R$}{R}}
\label{appendix:general_U}
Given $R\in SO(N^2-1)$, our task is to answer the follwing questions:
\begin{enumerate}
\item \label{q:in_SU_N}is $R\in\text{Ad}(SU(N))$?
\item \label{q:found_u}If the answer to question~\ref{q:in_SU_N} is yes,
  find $U\in SU(N)$ such that $R = \text{Ad}_U$.
\end{enumerate}

For any $U\in SU(N)$, let $U = Ve^{iH}V^\dagger$ be its eigendecomposition,
where $V\in SU(N)$, $H = \text{diag}(H_1,\ldots,H_N)$ and $\Tr H = 0$.
Furthermore, we can restrict $H$ such that for all $i, j$,
\begin{equation}
  \label{eq:h_ij}
  |H_i-H_j| \in (-2\pi, 2\pi].
\end{equation}
In fact, let $H_{i_+} = \max_i H_i$, $H_{i_-} = \min_i H_i$, if
$H_{i_+}-H_{i_-} > 2\pi$, we can redefine $H$ such that
\begin{equation}
  H_i\rightarrow\begin{cases}
  H_i - 2\pi & i = i_+, \\
  H_i + 2\pi & i = i_-, \\
  H_i & \text{otherwise}.
  \end{cases}
\end{equation}
Repeating the process and we can always find $H$ satisfying
Eq.~(\ref{eq:h_ij}).

Since $H$ commutes with any diagonal member of $\mathfrak{su}(N)$, then
the multiplicity of eigenvalue 0 of $\text{ad}_H$ should be at least $N-1$,
and the multiplicity of eigenvalue 1 of $\text{Ad}_U$ should be at least
$N-1$.

If the multiplicity of eigenvalue 1 of matrix $R$ is less than $N-1$, then
the answer to question~\ref{q:in_SU_N} is no and we stop here.

Otherwise, $R$ can be written in the form
\begin{equation}
  R = P\left(\left(\bigoplus_{l=1}^{N(N-1)/2}B(\phi_l)\right)
  \oplus I_{N-1}\right)P^T,
\end{equation}
where $P\in SO(N^2-1)$, $I_{N-1}$ is the $(N-1)\times(N-1)$ identity matrix,
and $B(\phi_l)$ are the $2\times2$ rotation matrix
\begin{equation}
  B(\phi) = \begin{bmatrix}\cos\phi&\sin\phi\\-\sin\phi&\cos\phi\end{bmatrix}.
\end{equation}
The logarithm of $R$ is
\begin{equation}
  \log R = P\left(\left(\bigoplus_{l=1}^{N(N-1)/2}\log B(\phi_l)\right)
  \oplus 0_{N-1}\right)P^T,
\end{equation}
where $0_{N-1}$ is a $(N-1)\times(N-1)$ matrix of zeros,
$\log B(\phi_l) = (\phi_l+2k_l\pi)iY$ with $Y$ being the second Pauli matrix.

If $R = \text{Ad}_U$ and $U = Ve^{iH}V^\dagger$, then
\begin{equation}
  -i\log R = (\text{Ad}_V)(\text{ad}_H)(\text{Ad}_V)^T.
\end{equation}
From Eq.~(\ref{eq:ad_h}) we conclude that each $\phi_l+2k_l\pi$ has to be
the difference of $H_i-H_j$ for some $(i,j)$.  Furthermore,
$-i\log R\in \mathfrak{su}(N)$.  Thus, if there exists
$\{k1,\ldots,k_{N(N-1)/2}\}$ such that $-2\pi <\phi_l+2k_l\pi\le2\pi$ and
\begin{equation}
  P\left(\left(\bigoplus_{l=1}^{N(N-1)/2}((\phi_l+2k_l\pi)Y)\right)
  \oplus0_{N-1}\right)P^T = \sum_{a=1}^{N^2-1}\theta_aF_a
\end{equation}
for some $\theta\in\mathbb R^{N^2-1}$ and $F_a = \text{ad}_{T_a}$, then
$R = \text{Ad}_U$, with $U = \exp(i\sum_a\theta_aT_a)$.

%%%%%%%%%%%%%%%%%%%%%%%%%%%%%%%%%%%%%%%%%%%%%%%%%%%%%%%%%%%%
% bibtex
\bibliographystyle{plain}
\bibliography{../BIBTEX/mybib}
%%%%%%%%%%%%%%%%%%%%%%%%%%%%%%%%%%%%%%%%%%%%%%%%%%%%%%%%%%%%
% end of document
\end{document}
